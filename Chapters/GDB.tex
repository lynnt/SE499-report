\chapter{GNU Debugger} \label{GDB}

\section{Introduction}
GBD is the GNU project debugger, which allows users to see what is going on
inside another program while it executes, or understand what a program was
doing at the moment it crashed.

GDB can do four main things to help catching bugs:
\begin{itemize}
\item Start your program by specifying anything that might affect the
behaviour.
\item Make your program stop on specified conditions.
\item Examine what happened and when the program has stopped.
\item Change things in the program, so users can experiment with correcting the
effects of one bug and other bugs.
\end{itemize}

\section{Stack Frames}

\section{Extensions}
GDB provides three mechanisms for extending the project. The first is
composition of GDB commands, the second solution is using Python scripting
language, and the third one is for defining new aliases of existing commands.

\section{Symbol Handling}
Symbols are a key part of GDB's operation. Symbols are variables, functions and
types. GDB has three types of symbol tables.
\begin{itemize}
    \item \textcolor{ForestGreen}{full symbol tables (symtabs)}: These contain the main information
        about symbols and addresses
    \item \textcolor{ForestGreen} {Partial symbol tables (psymtabs)}: These contain enough information to
        know when to read the corresponding part of the full symbol table.
    \item \textcolor{ForestGreen}{Minimal symbol tables (msymtabs)}: These contain information gleaned
        from non-debugging symbols.
\end{itemize}

Symbol information for a large program can be very large, and reading of all
these symbols can be a performance bottlenecks in GDB. This affects user
experience, so the solution is to construct partial symbol tables consisting of
only selected symbols, and then expand them to full symbol tables when
necessary.
A psymtab is constructed by doing a quick pass over the executable file's
debugging information.
