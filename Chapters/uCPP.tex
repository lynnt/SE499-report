\chapter{\uCPP} \label{uCPP}

\section{Introduction}
According to the \uCPPS annotated reference manual \cite{reference10}, \uCPPS
extends \CCS programming language. This language introduces new mechanism to
facilitate control flow, and adds new objects to enable lightweight concurrency
on uniprocessor and parallel execution on multiprocessor.

\iffalse
\section{Elementary Execution Properties}
\uCPPS was developed with three execution properties in mind:
\begin{itemize}
\item \textcolor{ForestGreen}{Thread}: is an execution of code that occurs
indepedently and possibly in concurrent with other execution. However, the
result of the execution is still sequential. The functionality of a thread is to
advance execution by changing its execution state. There are three basic states
of a thread. A thread is blocked when it is waiting for an occurrence of some
events. A thread is running if it is executing instructions on an actual
processor. A ready thread is when it is available for execution but is not
executing due to not being selected.
\item \textcolor{ForestGreen}{Execution State}: is the state information required
    to allow independent execution. Local data, local block, routine activations
    and current execution location, which is initialized to the starting point are considered to be part of an execution
    state. In particular, the local block and routine executions are maintained in a
    stack, and is the area where the local variables and execution is preserved
    when an execution state is inactive. Execution state is determined by a progarmming language, and is an elementary property of the semantics of a
language. Context switch is when control transfers from one execution state to
another.
\item \textcolor{ForestGreen}{Mutual Exclusion}: is an mechanism that allows an
    action to be performed on a resource without being interrupted by by other
    actions on the same resource. In a concurrent environment, mutual exclusion
    is required to maintain consistency of results. This mechanism cannot be
    trivially or efficiently implemented without appropriate programming
    language constructs.
\end{itemize}

\section{High-level Execution Constructs}
There are three elementary execution properties in \uCPP: thread, execution
state and mutual exclusion. These there properties are properties of an object. However, not all
combination of these properties are appropriate, and different combination
results into a different kind of objects.
\fi

\section{Tasks}
\textcolor{ForestGreen}{Task} behaves like a class object, but it maintains its own
thread and execution state, which is the state information required to allow
independent executions. A task provides mutual exclusion by default for its
public member methods. Tasks are used for communication between tasks and
concurency improvement since a task can context switch, which is a control transfer from
one execution state to another one, to another task waiting to be selected to
run on a processor if the current task needs to wait for an occurrence of an
event.

\section{\uCPPS Runtime Structure}
\uCPPS introduces two new runtime entities for controlling concurrent execution:
\begin{itemize}
    \item Cluster
    \item Virtual processor
\end{itemize}

\subsection{Cluster}
A cluster is a group of tasks and virtual processor (discussed next) that
execute tasks. The objective of a cluster is to control the amount of possible
parallelism, which is only feasible if there are many multiprocessors, among tasks.

At the start of a \uCPPS program, two clusters are created. One is system
cluster and the other cluster is user cluster. The system cluster uses a
processor that only performs management tasks such as error detection and
correction from user clusters if an execution in a user cluster results into
errors, and proper cleaning up when a shutdown occurs. A user cluster manages and executes user tasks on processors. The benefits of clusters
are maximized utilization of processors and miminization of runtime by
permitting many tasks to execute in a cluster. Due to the scheduling tasks model
in a cluster, load balacing tasks are performed to achieve the best result.
However, automatic load balancing between clusters does not exist, so manual
mitigration is performed by users.

\subsection{Virtual Processor}
A virtual processor is a software emulation of a processor that executes
threads. Kernel threads are used to implement a virtual processor, which are
scheduled for execution on a hardware processor by the underlying operating
system. The operating system distributes kernel threads across a number of
processors assuming that the program runs on a multiprocessor architecture. The
usage of kernel threads enables parallel execution in \uCPP.
