\chapter{\uCPP}
\CC is a popular programming language.

\textbf{\textcolor{red}{TODO}}: write a bit more about \CC

\textbf{\textcolor{red}{TODO}}: rephrase all these stuff in my own word and cite

According to the \uCPP annotated reference manual, \uCPP
for \CC programming language. This language introduces new ways to facilitate
control flow, and an addition of lightweight concurrency objects on
uniprocessor and parallel execution on multiprocessor on UNIX operating
system.

\section{Elementary Execution Properties}
\uCPP was developed with three execution properties in mind: thread,
execution state and mutual exclusion. First of all, the manual stated that
thread is an execution of code that occurs independently and potentially
concurrently with other executions in the same program, however, the actions
taken in the thread are still performed sequentially. Secondly, execution state
provides information about states, which allows independent execution. These
information can be local data, local block and routine activations. Finally,
mutual exclusion is a mechanism, in which enables an execution performed without
interruption by another action on a shared resource. In a concurrent system, mutual exclusion is required to guarantee consistent
generation
of results, and cannot be trivially or efficiently implemented without
appropriate programming language
constructs.
The first two properties represent the minimum needed to perform execution, and
seem to be fundamental in that
they are not expressible in machine-independent or language-independent ways.
For example, creating a new thread
requires creation of system runtime control information, and manipulation of
execution states requires machine specific
operations (modifying stack and frame pointers). The last property, while
expressible in terms of simple language
statements, can only be done by algorithms that are error-prone and inefficient,
e.g., Dekker-like algorithms, and
therefore, mutual exclusion must also be provided as an elementary execution
property, usually through special atomic
hardware instructions.

\section{High-level Execution Constructs}
\textbf{\textcolor{red}{TODO}}: rephrase all these stuff in my own word and cite
A programming language designer could attempt to provide these 3 execution
properties as basic abstractions in a
programming language [BLL88], allowing users to construct higher-level
constructs from them. However, some combinations
might be inappropriate or potentially dangerous. Therefore, all combinations are
examined, analyzing which
ones make sense and are appropriate as higher-level programming language
constructs. What is interesting is that
enumerating all combination of these elementary execution properties produces
many existing high-level abstractions
and suggests new ones.
The three execution properties are properties of objects. Therefore, an object
may or may not have a thread, may
or may not have an execution state, and may or may not have mutual exclusion.
Different combinations of these three
properties produce different kinds of objects. If an object has mutual
exclusion, this means that execution of certain
member routines are mutually exclusive of one another. Such a member routine is
called a mutual-exclusion member
(mutex member). In the situation where an object does not have the minimum
properties required for execution, i.e.,
thread and execution state, those of its caller are used.

Case 1 is an object, such as a free routine (a routine not a member of an
object) or an object with member routines
neither of which has the necessary execution properties, called a class object.
In this case, the caller’s thread and
execution state are used to perform execution. Since this kind of object
provides no mutual exclusion, it is normally
accessed only by a single thread. If such an object is accessed by several
threads, explicit locking may be required,
which violates a design requirement. Case 2 is like Case 1 but deals with the
concurrent-access problem by implicitly
ensuring mutual exclusion for the duration of each computation by a member
routine. This abstraction is a monitor
[Hoa74]. Case 3 is an object that has its own execution state but no thread.
Such an object uses its caller’s thread
to advance its own execution state and usually, but not always, returns the
thread back to the caller. This abstraction
is a coroutine [Mar80]. Case 4 is like Case 3 but deals with the
concurrent-access problem by implicitly ensuring
mutual exclusion; the name coroutine monitor has been adopted for this case.
Cases 5 and 6 are objects with a thread
but no execution state. Both cases are rejected because the thread cannot be
used to provide additional concurrency.
First, the object’s thread cannot execute on its own since it does not have an
execution state, so it cannot perform any
independent actions. Second, if the caller’s execution state is used, assuming
the caller’s thread can be blocked to
ensure mutual exclusion of the execution state, the effect is to have two
threads successively executing portions of a
single computation, which does not seem useful. Case 7 is an object that has its
own thread and execution state. Because
it has both a thread and execution state it is capable of executing on its own;
however, it lacks mutual exclusion.
Without mutual exclusion, access to the object’s data is unsafe; therefore,
servicing of requests would, in general,
require explicit locking, which violates a design requirement. Furthermore,
there is no performance advantage over
case 8. For these reasons, this case is rejected. Case 8 is like Case 7 but
deals with the concurrent-access problem by
implicitly ensuring mutual exclusion, called a task.
The abstractions suggested by this categorization come from fundamental
properties of execution and not ad hoc
6 CHAPTER 1. µC++ EXTENSIONS
decisions of a programming language designer. While it is possible to simplify
the programming language design by
only supporting the task abstraction [SBG+90], which provides all the elementary
execution properties, this would
unnecessarily complicate and make inefficient solutions to certain problems. As
will be shown, each of the nonrejected
abstractions produced by this categorization has a particular set of problems it
can solve, and therefore, each
has a place in a programming language. If one of these abstractions is not
present, a programmer may be forced to
contrive a solution for some problems that violates abstraction or is
inefficient.

\section{\uCPP Translator}
\uCPP has a translator that compiles $\mu$\CC code to normal \CC code.
\section{\uCPP Runtime Structure}
\textbf{\textcolor{red}{TODO}}: talk about clusters, processors, tasks and how
the thread mapping works in \uCPP

Beside the five new objects introduced by elementary properties, \uCPP also
introduces two more runtime entities for controlling concurrent execution.

\subsection{Virtual Processor}
A µC++ virtual processor is a “software processor” that executes threads. A
virtual processor is implemented by kernel
thread (normally created through a UNIX process) that is subsequently scheduled
for execution on a hardware
    processor by the underlying operating system. On a multiprocessor, kernel
    threads are usually distributed across the hardware processors and so some
    virtual processors are able to execute in parallel. µC++ uses virtual
    processors instead
    of hardware processors so that programs do not actually allocate and hold
    hardware processors. Programs can
    be written to run using a number of virtual processors and execute on a
    machine with a smaller number of hardware
    processors. Thus, the way in which µC++ accesses the parallelism of the
    underlying hardware is through an intermediate
    resource, the kernel thread. In this way, µC++ is kept portable across
    uniprocessor and different multiprocessor
    hardware designs.
    When a virtual processor is executing, µC++ controls scheduling of tasks on
    it. Thus, when UNIX schedules a
    virtual processor for a runtime period, µC++ may further subdivide that
    period by executing one or more tasks. When
    multiple virtual processors are used to execute tasks, the µC++ scheduling
    may automatically distribute tasks among
    virtual processors, and thus, indirectly among hardware processors. In this
    way, parallel execution occurs.
\subsection{Cluster}
A cluster is a collection of tasks and virtual processors (discussed next) that
execute the tasks. The purpose of a cluster
is to control the amount of parallelism that is possible among tasks, where
parallelism is defined as execution which
occurs simultaneously. Parallelism can only occur when multiple processors are
present. Concurrency is execution
that, over a period of time, appears to be parallel. For example, a program
written with multiple tasks has the potential
to take advantage of parallelism but it can execute on a uniprocessor, where it
may appear to execute in parallel
because of the rapid speed of context switching.
Normally, a cluster uses a single-queue multi-server queueing model for
scheduling its tasks on its processors (see
Chapter 10, p. 135 for other kinds of schedulers). This simple scheduling
results in automatic load balancing of tasks
on processors. Figure 2.1 illustrates the runtime structure of a µC++ program.
An executing task is illustrated by its
containment in a processor. Because of appropriate defaults for clusters, it is
possible to begin writing µC++ programs
after learning about coroutines or tasks. More complex concurrency work may
require the use of clusters. If several
clusters exist, both tasks and virtual processors, can be explicitly migrated
from one cluster to another. No automatic
load balancing among clusters is performed by µC++.
When a µC++ program begins execution, it creates two clusters: a system cluster
and a user cluster. The system
cluster contains a processor that does not execute user tasks. Instead, the
system cluster handles system-related operations,
such as catching errors that occur on the user clusters, printing appropriate
error information, and shutting
down µC++. A user cluster is created to contain the user tasks; a staring task
is created in the user cluster, called main
with type uMain, which calls routine main. Having all tasks execute on the one
cluster often maximizes utilization of
processors, which minimizes runtime. However, because of limitations of the
underlying operating system or because
of special hardware requirements, it is sometimes necessary to have more than
one cluster. Partitioning into clusters
must be used with care as it has the potential to inhibit parallelism when used
indiscriminately. However, in some situations
partitioning is essential, e.g., on some systems concurrent UNIX I/O operations
are only possible by exploiting
the clustering mechanism.
