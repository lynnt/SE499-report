\chapter{\uCPP} \label{uCPP}

\section{Introduction}
\uCPPS \cite{reference10} extends the \CCS programming language with new
mechanisms to
facilitate control flow, and adds new objects to enable lightweight concurrency
on uniprocessor and parallel execution on multiprocessor computers. Concurrency has tasks
that can context switch, which is a control transfer from one execution state to
another that is different from the routine call. Tasks are selected to run on a
processor from a ready queue of available tasks, and tasks may need to wait for an occurrence of an event.

\section{Tasks}
A \textcolor{ForestGreen}{task} behaves like a class object, but it maintains its own
thread and execution state, which is the state information required to allow
independent execution. A task provides mutual exclusion by default for calls to its
public members. Public members allow communication among tasks.

\section{\uCPPS Runtime Structure}
\uCPPS introduces two new runtime entities for controlling concurrent execution:
\begin{itemize}
    \item Cluster
    \item Virtual processor
\end{itemize}

\subsection{Cluster}
A cluster is a group of tasks and virtual processors (discussed next) that
execute tasks. The objective of a cluster is to control the amount of possible
parallelism mong tasks, which is only feasible if there are many
multiprocessors.

At the start of a \uCPPS program, two clusters are created. One is the system
cluster and the other is the user cluster. The system cluster has a
processor that only performs management work such as error detection and
correction from user clusters, if an execution in a user cluster results in errors,
and cleans up at shutdown. The user cluster manages and
executes user tasks on its processors. The benefits of clusters
are maximized utilization of processors and miminization of runtime through a scheduler that is appropriate for a particular workload. Tasks and virtual
processors may be migrated among clusters.

\subsection{Virtual Processor}
A virtual processor is a software emulation of a processor that executes
threads. Kernel threads are used to implement a virtual processor, which are
scheduled for execution on a hardware processor by the underlying operating
system. The operating system distributes kernel threads across a number of
processors assuming that the program runs on a multiprocessor architecture. The
usage of kernel threads enables parallel execution in \uCPP.
