\chapter{Extending GDB for \uCPP}

\textcolor{red}{TODO}
\begin{itemize}
\item Summary design at a high level
\item Describe what design does and how it works along with a scenario for its
use case
\item Use subsections to guide through details of design
\item Start with a section with block diagram that shows major functions/layout
of design
\item Evaluate how the solution works by evaluating the design against
requirements you outlined in problem definition section.
\item State how the project should move ahead
\end{itemize}

\section{Design Constraints}
As mentioned in chapter \ref{GDB}, there are many ways to extend GDB. However,
there are a few design constraints on the selected mechanism. Extension
mechanism for \uCPPS should be simple, easy to implement and versatile.

Python extensions satifies all the given criteria. In particular, GDB provides a more extensive API for Python to work with. Additionally, Python
is a scripting language with built-in data structures and functions, which make
it easy to work with.

As mentioned in chapter \ref{uCPP}, there are concepts such as clusters and
virtual processors in \uCPP. It is useful to have GDB commands to list
all the existing clusters and virtual processors. This provides developers
a mechanism to understand what is going on in the system. Furthermore, when users
press Ctrl-C or a failure occurs when running a \uCPPS program, one does not
know which task the program is in, and which cluster that task falls into. Thus,
having a way to look at this information and proceed from there is tremedously
useful. This results into requirement for four commands. One is for listing out
all the available clusters, one for tasks, one for processors and finally one
for listing out all tasks available within a particular cluster.

\section{Design Implementation}
Seven user-defined commands are implemented as an extension in GDB for \uCPP.
