\chapter{Extending GDB for \uCPP}
\section{Design Constraints}
As mentioned in chapter \ref{GDB}, there are many ways to extend GDB. However,
there are a few design constraints on the selected mechanism. Extension
mechanism for \uCPPS should be simple, easy to implement and versatile.

\textcolor{red}{TODO:} rewrite this to talk about flexibility in commands

Regardless of which mechanism selected for extending GDB, all the functions
implemented should maintain the same functionality. In addition to functional
requirements, user usability is a required non-functional requirement for this
project. The justification for this requirement is reduction time in
learning and using the extensions.

\section{Design Implementation}
First of all, after doing resarching about different approaches and the requirements for new
commands, Python API satifies all the given criteria above. In particular, GDB provides a more extensive API for Python to work with. Additionally, Python
is a scripting language with built-in data structures and functions, which make
it easy to work with.

The next task for extending GDB is writing new user-defined commands that allow
users to switch from one task to another task. This means that there needs to be at
least two commands. The first command is for switching to the new task and let
call it \verb|push_task|, and another command
is for switching back to the previous task, from which the current task comes
from, and let name this command \verb|pop_task|.

Prior to implement \verb|pushtask| or \verb|poptask| command, many questions come up about how
these command should work.
\begin{itemize}
\item What format of the argument of the new task should it be for command
\verb|pushtask|?
\item Should \verb|pushtask| command allow users to pass the address of the desired task or can users
pass in the identification of the cluster the task is in along with the unique
identification of the task with respect to the cluster?
\item Should \verb|pushtask| command save context of the current task before it switches to
another task for every switch or should the command save only the context of the
originating task?
\end{itemize}

As a result, there are two variations of \verb|pushtask| command to provide both
user usability and flexibility. The first option is called \verb|pushtask| and requires
only one argument, which is the address of the task a user wants to switch to.
The second variation is called \verb|pushtask_id| and requires both the index of
the cluster and the index of the task they would like
to switch to as arguments. Core functionality of these two commands is
essentially the same. The only difference is that the \verb|pushtask_id| command
needs to iterate through the data structure that stores all the clusters to find
the right cluster baed on its index, and from the instance of the cluster, the
command looks for the instance of the respective task. From there, the second
command invokes the function that both share in common. Furthermore, a decision
is made to store context information for every context switching, this means
that \verb|pushtask| command needs to perform actions to store these information every time it is
invoked. Each task has a copy of \verb|uContext_t|, which stores context
information such as stack pointer, frame pointer. These pointers are then stored
in convenience variables for every level of redirection. Similarly, the command
retrieves these context information from the task a user wants to switch to, and
set the equivalent registers to these values. In particular, value of stack
pointer is set in \verb|rsp| register, value of frame pointer is set in \verb|rbp|
register, and finally value of program counter is set in \verb|pc| register.
Normally, register values are relative to the current stack frame assuming that
all the stack frames farther exited and their registers are restored, therefore, in
order to see the true value of hardware registers, innermost frame must be
selected, which is frame 0\cite{reference}. However, it is possible to not to be in frame 0, so prior to setting these values,
the command switches back to the innermost (currently executing) frame first.

Finally, the implementation of \verb|poptask| is more straightforward. This
command goes back to the previous task, however, this may not be the original
task that initiates the task switching process. If the current context is
already in the starting task, the command does nothing. Otherwise, \verb|poptask| command first switches back to the innermost frame, and then retrieve the
address of the last task it was switched from, and set the same registers as
\verb|pushtask| command did but to the values of the last task's context
information.
