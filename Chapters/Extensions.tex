\chapter{Extending GDB for \uCPP}
When a concurrent program deadlocks, looking at the task call-stack can locate
the resource and the blocking cycle that resulted in the deadlock. Therefore,
a reason for having a task tour is to display the call-stack of each task to
know where it is executing and what values it is currently computing. This leads
to the development of extending GDB.

\section{Design Constraints}
As mentioned in Chapter \ref{GDB}, there are several ways to extend GDB. However,
there are a few design constraints on the selected mechanism. All the functions
implemented should maintain similar functionality to existing GDB commands. In addition to functional
requirements, usability and flexibility are additional requirements for this
project. These final requirements enable developers to be productive quickly
and do more with the extensions. The extensions created for \uCPPS are simple
to use and versatile.

\section{Existing User-defined GDB Commands}
\subsection{\uCPPS source code example}
Listing \ref{uCPP-src-code} shows a \uCPPS program that implicitly creates two
clusters, system and user, explicitly creates three processors plus one
implicit for system and user cluster. Additionally, the program explicitly creates ten tasks on the
user cluster, plus the implicit system and processor tasks on the system cluster, and
processor tasks on the user cluster.

\begin{lstlisting}[caption={\uCPPS source code used for GDB commands},
label={uCPP-src-code}, basicstyle=\small]
_Task T
{
    string name;

    void a(int param)
    {
        while (1);
    }

    void main()
    {
        a(5);
    }
    public:
        T(const int tid)
        {
            name = "T" + std::to_string(tid);
            setName(tid.c_str());
        }
};

int main()
{
    // set the number of processors
    uProcessor procs[3];
    const int numTasks = 10;
    T* tasks[numTasks];

    // initialize tasks T
    for (int id = 0; id < numTasks; id += 1)
    {
        tasks[i] = new T(id);
    }

    // clean up all tasks
    for (int id = 0; id < numTasks; id += 1)
    {
        delete tasks[i];
    }
}
\end{lstlisting}

\subsection{Listing all clusters in a \uCPPS program}
\begin{lstlisting}[caption={clusters command}, label={clusters-command}]
(gdb) clusters
        Name            Address
    systemCluster       0x655280
    userCluster         0x7c4280
\end{lstlisting}
Listing \ref{clusters-command} shows a list of all clusters available in the
program
along with their associated address.

\subsection{Listing all processors in a cluster}
\begin{lstlisting}[caption={\text{cluster\_procs} command}, label={cluster-procs}]
(gdb) processors userCluster
    Address    PID         Preemption      Spin
    0x7c6bf0    8396928     10              1000
    0x8c3b60    9453696     10              1000
    0x8c3d20    9977984     10              1000
    0x8c3ee0    10506368    10              1000
\end{lstlisting}
In listing \ref{cluster-procs}, the command \verb|processors| expects one
argument. This command shows
all the processors in a cluster. In particular, this example shows that there
are four processors in \verb|userCluster|, where each has an address,
its PID, its premption priority and spin, which is how many times a processor
checks its associated cluster's ready queue for an available task to execute
before the processor blocks.

\subsection{Listing all tasks in all clusters}
\begin{lstlisting}[caption={\text{tasks} command}, label={tasks},
basicstyle=\small]
(gdb) task
        Cluster Name        Address
        systemCluster       0x655280
    ID      Task Name            Address         State
    0       uProcessorTask      0x6c39b0        uBaseTask::Blocked
    1       uSystemTask         0x783ef0        uBaseTask::Blocked
        Cluster Name        Address
        userCluster         0x7c4280
    ID      Task Name            Address         State
    0       uProcessorTask      0x806e70        uBaseTask::Blocked
    ...
    6       T0                  0xa49830        uBaseTask::Running
    7       T1                  0xa89bb0        uBaseTask::Ready
    8       T2                  0xac9f30        uBaseTask::Ready
    ...
    15      T9                  0xc8b7b0        uBaseTask::Ready
\end{lstlisting}
In listing \ref{tasks}, the command \verb|task| does not require any
arugment since it lists out all the known tasks in a \uCPPS program at this
point of the snapshot of the execution.

\subsection{Listing all tasks in a cluster}
\begin{lstlisting}[caption={\text{cluster\_tasks} command}, label={cluster-tasks}]
(gdb) task systemCluster
    ID          Name            Address         State
    0       uProcessorTask      0x654200        uBaseTask::Blocked
    1       uSystemTask         0x7c3280        uBaseTask::Blocked
\end{lstlisting}
In listing \ref{cluster-tasks}, the command \verb|task| requires one argument
that is the name of a cluster. In this example, this command lists
all the name of the tasks in a cluster called \verb|systemCluster|.
Additionally, this version of the command \verb|task| also provides the associated address
for each task and its state.

\section{Design Implementation}
The GDB Python API satifies all the given criteria above, after examining the different solutions. In particular, GDB provides an extensive API for
the programming language Python compared to other solutions. Furthermore, Python is a scripting language with built-in
data structures and functions that enables the development of more complex
operations and saves time on development.

The next extension for displaying information is writing new commands that allow
stepping from one \uCCS task to another. Two switching approaches are
provided: structured and free form. The structured form remembers the task tour in a
LIFO way, while the free form does not remember the task tour. This semantics means
at least two commands are needed for the structured form and the free form.
Taking the idea of overloading in programming languages, the structured
commands for switching from the current task to a different task overload the
command \verb|task| with two  arguments, and
for switching back to the previous task is \verb|poptask|.

Prior to executing the overloaded \verb|task| or \verb|poptask| stepping, the
task argument has to be resolved. There are two ways to designate a task: its
absolute address or relative id within a cluster. For instance, to switch from
any task to task \verb|T1| seen in listing \ref{tasks}, \verb|task 0xa89bb0| can use an absolute address (0xa89bb0), or \verb|task userCluster 7| can use the relative id (7) with respect to userCluster.

As a result, both the structured and free form approaches are supported, which
adds higher degree of freedom to users. Core functionality of these approaches
is essentially the same. To implement the task tour, it is necessary
to store the context information for every context switching. This means that
the \verb|task| function in listing \ref{pushtask} needs to perform action to
store this information every time it is invoked.

In listing \ref{pushtask-code}, each task points to a \verb|uContext_t| data
structure, which stores the context
information such as stack pointer, frame pointer. These pointers are then
stored in an instance of the named tuple \verb|stackInfo| for every level of
redirection. The tuple also stores information about the program counter that
is calculated from the address of the \verb|uSwitch| Assembly routine because a task always ends up in \verb|uSwitch| when its state is blocked. Similarly, the command
also retrieves these context information but from the task that a user wants to switch to, and
sets the equivalent registers to the new values. In particular, the value of the stack
pointer is set in \verb|rsp| register, the value of the frame pointer is set in \verb|rbp|
register, and finally the value of the program counter is set in \verb|pc| register.
For every new function call, a new stack frame is created and the values of all
the registers are changed for that frame. Therefore, in
order to see the true value of hardware registers, innermost frame that is
frame-0 must be selected \cite{reference11}. However, it is possible to not be in frame-0, so prior to setting these values,
the command must switch back to the innermost (currently executing) frame first.

\begin{lstlisting}[language=Python, caption={Abridged \text{push\_task} source
code}, label={pushtask-code}, fontstyle=\small]
# get GDB type of uContext_t*
uContext_t_ptr_type = gdb.lookup_type('UPP::uMachContext::uContext_t').pointer

# retrieve the context object from a task
# and cast it to the type uContext_t*
task_context = task['context'].cast(uContext_t_ptr_type)

# the offset where sp would be starting from uSwitch function
sp_address_offset = 48
# lookup the value of stack pointer (sp), frame pointer (fp),
# program counter (pc)
xsp = task_context['SP'] + sp_address_offset
xfp = task_context['FP']
if not gdb.lookup_symbol('uSwitch'):
    print('uSwitch symbol is not available')
    return

# This value is calculated here beacuse we always here when the task is in
blocked state
xpc = get_addr(gdb.parse_and_eval('uSwitch').address + 28, 'PC')
# must switch back to frame-0 to set 'pc' register with
# the value of xpc
gdb.execute('select-frame 0')

# retrieve register values and push sp, fp, pc into a global stack
global STACK
sp = gdb.parse_and_eval('$sp')
fp = gdb.parse_and_eval('$fp')
pc = gdb.parse_and_eval('$pc')
stack_info = StackInfo(sp = sp, fp = fp, pc = pc)
STACK.append(stack_info)

# update registers for a new task
gdb.execute('set $rsp={}'.format(xsp))
gdb.execute('set $rbp={}'.format(xfp))
gdb.execute('set $pc={}'.format(xpc))
\end{lstlisting}

Finally, the implementation of \verb|poptask| is more straightforward. This command goes back to the previous task, however, this may not be the original
task that initiates the task switching process. If the current context is
already in the starting task, the command does nothing. Otherwise,
\verb|poptask| command first switches back to the innermost frame, and then
retrieves the
address of the last task it was switched from, and sets the same set of registers as
\verb|pushtask| command did but to the values of the last task's context
information. An example of this command can be seen in listing \ref{poptask}.

In addition to implementing these new GDB commands, additional hooks are added
to existing built-in GDB commands that continue the execution of a program in any form. These hooks
call a new command called \verb|reset| prior to executing the command to enable
continuation of a program to ensure that
the program's context is automatically switched back to the context of the task
that initiates the first
context switch.
\begin{lstlisting}[caption={\text{pushtask} command}, label={pushtask}]
(gdb) task 0xac9f30
\end{lstlisting}

\begin{lstlisting}[caption={Context switch version of task command}, label={pushtask-id}]
(gdb) bt
#0  0x00007ffff6d79826 in __GI___sigsuspend (set=set@entry=0x73f850)
    at ../sysdeps/unix/sysv/linux/sigsuspend.c:30
#1  0x000000000042f2ef in uCluster::processorPause (
    this=0x655280 <uKernelModule::systemClusterStorage>)
    at /usr/local/u++-7.0.0/src/kernel/uCluster.cc:120
#2  0x000000000041b5f1 in UPP::uProcessorKernel::main (this=0x703db0)
    at /usr/local/u++-7.0.0/src/kernel/uProcessor.cc:678
#3  0x00000000004257d6 in UPP::uMachContext::invokeCoroutine (This=...)
    at /usr/local/u++-7.0.0/src/kernel/uMachContext.cc:89
#4  0x0000000000000000 in ?? ()
(gdb) task userCluster 7
(gdb) bt
#0  uSwitch () at /usr/local/u++-7.0.0/src/kernel/uSwitch-x86_64.S:64
#1  0x00000000004287dc in uBaseCoroutine::taskCxtSw (this=0x8c3ef8)
    at /usr/local/u++-7.0.0/src/kernel/uBaseCoroutine.cc:146
#2  0x000000000042cd3e in UPP::uProcessorKernel::scheduleInternal (this=<optimized out>,
    task=<optimized out>) at /usr/local/u++-7.0.0/src/kernel/uProcessor.cc:291
#3  0x0000000000429f76 in uBaseTask::uYieldInvoluntary (this=<optimized out>)
    at /usr/local/u++-7.0.0/src/kernel/uBaseTask.cc:361
#4  0x000000000041e4e4 in uKernelModule::rollForward (inKernel=inKernel@entry=false)
    at /usr/local/u++-7.0.0/src/kernel/uC++.cc:1111
#5  0x000000000042bfc3 in UPP::uSigHandlerModule::sigAlrmHandler (sig=<optimized out>,
    sfp=<optimized out>, cxt=0xac5380) at /usr/local/u++-7.0.0/src/kernel/uSignal.cc:308
#6  <signal handler called>
#7  T::a (this=0xa89bb0) at utils.cpp:7
#8  0x000000000041ca60 in T::main (this=0xa89bb0) at utils.cpp:10
#9  0x0000000000425aac in UPP::uMachContext::invokeTask (This=...)
    at /usr/local/u++-7.0.0/src/kernel/uMachContext.cc:140
#10 0x0000000000000000 in ?? ()
\end{lstlisting}

\begin{lstlisting}[caption={\text{poptask} command}, label={poptask}]
(gdb) bt
#0  0x00007ffff6d79826 in __GI___sigsuspend (set=set@entry=0x73f850)
    at ../sysdeps/unix/sysv/linux/sigsuspend.c:30
#1  0x000000000042f2ef in uCluster::processorPause (
    this=0x655280 <uKernelModule::systemClusterStorage>)
    at /usr/local/u++-7.0.0/src/kernel/uCluster.cc:120
#2  0x000000000041b5f1 in UPP::uProcessorKernel::main (this=0x703db0)
    at /usr/local/u++-7.0.0/src/kernel/uProcessor.cc:678
#3  0x00000000004257d6 in UPP::uMachContext::invokeCoroutine (This=...)
    at /usr/local/u++-7.0.0/src/kernel/uMachContext.cc:89
#4  0x0000000000000000 in ?? ()
....
(After context switched to task 7 in userCluster)
(gdb) bt
#0  uSwitch () at /usr/local/u++-7.0.0/src/kernel/uSwitch-x86_64.S:64
#1  0x00000000004287dc in uBaseCoroutine::taskCxtSw (this=0x8c3ef8)
    at /usr/local/u++-7.0.0/src/kernel/uBaseCoroutine.cc:146
#2  0x000000000042cd3e in UPP::uProcessorKernel::scheduleInternal (this=<optimized out>,
    task=<optimized out>) at /usr/local/u++-7.0.0/src/kernel/uProcessor.cc:291
#3  0x0000000000429f76 in uBaseTask::uYieldInvoluntary (this=<optimized out>)
    at /usr/local/u++-7.0.0/src/kernel/uBaseTask.cc:361
#4  0x000000000041e4e4 in uKernelModule::rollForward (inKernel=inKernel@entry=false)
    at /usr/local/u++-7.0.0/src/kernel/uC++.cc:1111
#5  0x000000000042bfc3 in UPP::uSigHandlerModule::sigAlrmHandler (sig=<optimized out>,
    sfp=<optimized out>, cxt=0xac5380) at /usr/local/u++-7.0.0/src/kernel/uSignal.cc:308
#6  <signal handler called>
#7  T::a (this=0xa89bb0) at utils.cpp:7
#8  0x000000000041ca60 in T::main (this=0xa89bb0) at utils.cpp:10
#9  0x0000000000425aac in UPP::uMachContext::invokeTask (This=...)
    at /usr/local/u++-7.0.0/src/kernel/uMachContext.cc:140
#10 0x0000000000000000 in ?? ()
(gdb) poptask
#1  0x000000000042f2ef in uCluster::processorPause (this=0x73f7a8)
    at /usr/local/u++-7.0.0/src/kernel/uCluster.cc:120
120     /usr/local/u++-7.0.0/src/kernel/uCluster.cc: No such file or directory.
(gdb) bt
#0  0x00007ffff6d79826 in __GI___sigsuspend (set=set@entry=0x73f850)
    at ../sysdeps/unix/sysv/linux/sigsuspend.c:30
#1  0x000000000042f2ef in uCluster::processorPause (this=0x73f7a8)
    at /usr/local/u++-7.0.0/src/kernel/uCluster.cc:120
#2  0x000000000041b5f1 in UPP::uProcessorKernel::main (this=0x703db0)
    at /usr/local/u++-7.0.0/src/kernel/uProcessor.cc:678
#3  0x00000000004257d6 in UPP::uMachContext::invokeCoroutine (This=...)
    at /usr/local/u++-7.0.0/src/kernel/uMachContext.cc:89
#4  0x0000000000000000 in ?? ()
\end{lstlisting}

\section{Result}
The current implementation successfully allows users to display snapshot of the
current execution with respect to clusters, processors, and tasks. With this
information it is possible to tour the call stacks of the tasks to see execution
locations and data values. Additionally, users are allowed to continue the execution where the
program last pauses assuming that the program has not crashed. The continuation
of execution can be done automatically by the extensions by running any already
existing GDB commands such as \verb|continue| or a user can manually
perform the reverse of context switch using the command \verb|poptask|.
