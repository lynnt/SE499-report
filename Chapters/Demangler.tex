\chapter{\CFAS Demangler} \label{demangler}

\section{Design Constraints}
While \CFAS is a translator for additional features that C does not support, all
the code during the compilation phase is eventually compiled down to C code.
The resulting executable file marks the DWARF tag \verb|DW_AT_language| with the
fixed hexadecimal value for the C language. Because it is possible to have one frame in C code and another
frame in Assembly code, GDB encodes a language flag for each frame. \CFAS adds
to this list, as it is essential to know when a stack frame contains mangled
names versus C and assembler unmangled names. Thus, the first requirement is
making GDB understands \CFAS as distinct a source language.

Next, there is a DWARF reader component in GDB since most GDB targets use the DWARF
format. The DWARF reader sets all the appropriate information read from object or
executable files in DIE structures within GDB as mentioned in Chapter
\ref{GDB}. However, GDB currently does not understand the DWARF language code
assigned to the
language \CFA, so another requirement is updating the DWARF reader in GDB to
recognize \CFA.

Additionally, when a user enters a name into GDB, GDB needs to lookup if the
name exists in the program. However, different languages may have different
hierarchical structure for dynamic scope, so an implementation for nonlocal
symbol lookup is required.

\section{Design Implementation}
Most of the implementation work discussed below is from reading GDB's internals
wiki page and understanding how other languages are supported in GDB \cite{reference5}.

A new entry is added to GDB's list of language definition in
\verb|gdb/defs.h|. Hence, a new instance of type \verb|struct language_def|
must be created to add a language definition of \CFAS. This instance is the
entry point for new functions that are only applicable to \CFA. These new
functions are invoked by GDB during debugging if there are operations that
are applicable specifically to \CFA. For instance, \CFAS can implement its
own symbol lookup function for non-local variables because \CFAS can have a
different scope hierarchy. The final step for registering \CFAS in GDB, as a new
source language, is adding the instance of type \verb|struct language_def| into
the list of language definitions, which can be found in
\verb|gdb/language.h|. This setup is shown in listing \ref{cfa-lang-def}.

\begin{lstlisting}[style=C++, caption={Language definition declaration for
\CFA}, label={cfa-lang-def}, basicstyle=\small]
// In gdb/language.h
extern const struct language_defn cforall_language_defn;

// In gdb/language.c
static const struct language_defn *languages[] = {
    &unknown_language_defn,
    &auto_language_defn,
    &c_language_defn,
    ...
    &cforall_language_defn,
    ...
 }

 // In gdb/cforall-lang.c
extern const struct language_defn cforall_language_defn =
{
    "cforall",                      /* Language name */
    "CForAll",                      /* Natural name */
    language_cforall,
    range_check_off,
    case_sensitive_on,
    ...
    cp_lookup_symbol_nonlocal,      /* lookup_symbol_nonlocal */
    ...
    cforall_demangle,               /* Language specific demangler */
    cforall_sniff_from_mangled_name,
    ..
}
\end{lstlisting}

The next step is updating the DWARF reader, so the reader can translate the DWARF code to an enum value defined
above. However, this assumes that the language has an assigned language code.
The language code is a hexadecimal literal value assigned to a particular
language, which is maintained by GCC. For \CFA, the hexidecimal value
\verb|0x0025| is added to \verb|include/dwarf2.h| to denote \CFA, which is shown
in listing \ref{cfa-dwarf}.

\begin{lstlisting}[style=C++, caption={DWARF language code for \CFA},
label={cfa-dwarf}, basicstyle=\small]
// In include/dwarf2.h
/* Source language names and codes.  */
enum dwarf_source_language
{
    DW_LANG_C89 = 0x0001,
    ...
    DW_LANG_CForAll = 0x0025,
}
\end{lstlisting}

Once the demangler implementation goes into the \verb|libiberty| directory along with
other demanglers, the demangler can be called by updating the language
definition defined in listing \ref{cfa-lang-def} with the entry point of the
\CFAS demangler, and adding a check if the current demangling style is
\verb|CFORALL_DEMANGLING| as seen in listing \ref{cfa-demangler}. GDB then
automatically invokes this \CFAS demangler
when GDB detects the source language is \CFA. In addition to the automatic
invocation of the demangler, GDB provides an option
to manually set which demangling style to use in the command line interface.
This option can be turned on for \CFAS in GDB by adding a new enum value for \CFAS in
the list of demangling styles along with setting the appropriate mask for this
style in \verb|include/demangle.h|. After doing this step, users can now choose
if they want to use the \CFAS demangler in GDB by calling \verb|set demangle-style <language>|, where the language name is defined as the
preprocessor macro \verb|CFORALL_DEMANGLING_STYLE_STRING| in listing
\ref{cfa-demangler-style}.

\begin{lstlisting}[style=C++, caption={libiberty setup for the \CFAS demangler},
label={cfa-demangler}, basicstyle=\small]
// In libiberty/cplus-dem.c
const struct demangler_engine libiberty_demanglers[] =
{
    {
        NO_DEMANGLING_STYLE_STRING,
        no_demangling,
        "Demangling disabled"
    }
    ,
    ...
    {
        CFORALL_DEMANGLING_STYLE_STRING,
        cforall_demangling,
        "Cforall style demangling"
    },
    ...
}
...
char *
cplus_demangle (const char *mangled, int options)
{
    ...
    /* The V3 ABI demangling is implemented elsewhere.  */
    if (GNU_V3_DEMANGLING || RUST_DEMANGLING || AUTO_DEMANGLING)
    {
        ...
    }
    ...
    if (CFORALL_DEMANGLING)
    {
        ret = cforall_demangle (mangled, options);
        if (ret) {
            return ret;
        }
    }
}
\end{lstlisting}

\begin{lstlisting}[style=C++, caption={Setup \CFAS demangler style},
label={cfa-demangler-style}, basicstyle=\small]
// In gdb/demangle.h
#define DMGL_CFORALL (1 << 18)
...
/* If none of these are set, use 'current_demangling_style' as the default. */
#define DMGL_STYLE_MASK
(DMGL_AUTO|DMGL_GNU|DMGL_LUCID|DMGL_ARM|DMGL_HP|DMGL_EDG|DMGL_GNU_V3
|DMGL_JAVA|DMGL_GNAT|DMGL_DLANG|DMGL_RUST|DMGL_CFORALL)
...
extern enum demangling_styles
{
    no_demangling = -1,
    unknown_demangling = 0,
    ...
    cforall_demangling = DMGL_CFORALL
} current_demangling_style;
...
#define CFORALL_DEMANGLING_STYLE_STRING          "cforall"
...
#define CFORALL_DEMANGLING (((int) CURRENT_DEMANGLING_STYLE) & DMGL_CFORALL)
\end{lstlisting}

However, the setup for the \CFAS demangler above does not demangle mangled
symbols during symbol-table lookup while the program is in progress. Therefore, additional work needs to be done in
\verb|gdb/symtab.c|. Prior to looking up the symbol, GDB attemps to demangle
the name of the symbol, which can either be mangled or unmangled name, to see
if it can detect the language, and select the appropriate demangler to demangle the symbol. This work
enables invocation of the \CFAS demangler during symbol lookup.
\begin{lstlisting}[style=C++, caption={\CFAS demangler setup for symbol lookup},
label={cfa-symstab-setup}, basicstyle=\small]
// In gdb/symtab.c
const char *
demangle_for_lookup (
    const char *name,
    enum language lang,
    demangle_result_storage &storage
    )
{
    /* If we are using C++, D, or Go, demangle the name before doing a
    lookup, so we can always binary search.  */
    if (lang == language_cplus)
    {
        char *demangled_name = gdb_demangle (name, DMGL_ANSI | DMGL_PARAMS);
        if (demangled_name != NULL)
            return storage.set_malloc_ptr (demangled_name);
    }
    ...
    else if (lang == language_cforall)
    {
        char *demangled_name = cforall_demangle (name, 0);
        if (demangled_name != NULL)
            return storage.set_malloc_ptr (demangled_name);
    }
    ...
}
\end{lstlisting}

\section{Result}
The addition of hooks throughout GDB enables the invocation of the new \CFAS
demangler during symbol lookup and during the usage of binutils tools such as
objdump, nm. Additionally, these binutils tools also understand \CFAS because
of the addition of the \CFAS language code. However, as the
language develops, symbol lookup for non-local variables must be implemented to
produce the correct output.
