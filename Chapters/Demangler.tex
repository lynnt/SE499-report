\chapter{\CFAS Demangler} \label{demangler}

\section{Design Constraints}
Since \CFAS is a translator for additional features that C does not support, all
the code during the compilation phase is eventually compiled down to C code.
The resulting executable file marks the tag DW\_AT\_language with a
fixed hexadecimal value of the language that the compiler determines at compile
time. Furthermore, it is possible to have one frame is in C code, and another
frame in Assembly code, so GDB encodes a language flag for each frame. This
results into a constraint in implementing the demangler. The first requirement is
making GDB understands \CFAS as a source language.

Next, because GDB relies on debug information written in DWARF format, DWARF
reader is a component in GDB, which parses the DWARF format from the object or
executable file to retrieve and set
appropriate information in the DIE structure within GDB as mentioned in chapter
\ref{GDB}. However, the encoded value is a hexadecimal value, and GDB needs to
know how to map that value into the equivalent format that GDB understands.
Another requirement is updating the DWARF reader in GDB to recognize \CFA.

Additionally, before the process demangling starts,

\section{Design Implementation}
\CFAS As A New Source Language in GDB

Adding \CFAS to DWARF reader

Symbol lookup for \CFAS

Adding new demangler
