\chapter{Introduction} \label{introduction}
Computer programming languages provide humans a means to instruct computers to
perform a particular task. New programming languages are
invented to simplify the task, or provide additional features, enhance
performance, and improve developer productivity.

A crucial companion tool to a programming language is a debugger. A debugger is a productivity tool to aid developers in testing
and finding bugs in a program. By definition, a debugger executes
any program written in one of a supported set of languages and allows developers
to stop, monitor and examine state in the program for further investigation.

Specifically, this report talks about how to add GNU Debugger (GDB) support for the
concurrent programming-languages \uCPPS and \CFA.
Both languages provide an $M$:$N$ concurrency model, where $M$ user-level threads
execute on $N$ kernel-level threads.
Often debuggers either do not know about concurrency or only provide a simple
understanding of concurrency provided by the operating system (kernel threads).
For \CFA, new hooks are also added to allow GDB to understand that
\CFAS is a new source language that requires invocation of a demangler for
variable and function names.

Because \uCPPS is a translator, all the code written in \uCPPS is eventually
compiled down to \CCS code. This transformation gives \uCPPS an advantage with
GDB because GDB already understands \CC. However, since \uCPPS introduced new objects
and high-level execution constructs into the language, GDB does not understand
these objects or the runtime environment. One objective of this
project is to write new GDB extensions to understand concurrency among tasks, a new high-level execution construct that is discussed more in Chapter \ref{uCPP}.

Additionally, if a programming language provides overloading functionality,
which allows functions or variables in the same scope with the
same identifier, then each of these entities must be assigned a unique name, otherwise,
there are name collisions.

However, uniquely numbering (naming) overloads is impossible with separate
compilation, because the compiler does not have access to all translation
units. Therefore, it is necessary to adopt a scheme related to the overloading
mechanism for unique naming. For example, if a language uses the number and
types of function parameters to disambiguate function names, than the number
and parameter types are encoded into the name:
\begin{lstlisting}[basicstyle=\normalfont\tt]
void f( int i, double d, char c );  // f_3_i_d_c
void f( int i, char c );            // f_2_i_c
\end{lstlisting}
Here, the mangled names for \lstinline@f@ contain the number of parameters and a code for
each parameter type. For a complex type-system, the type codes become
correspondingly complex, e.g., a generic structure. These names are now unique
across all translation units.

Unfortunately, a debugger only has access to the mangled names in a compiled
translation units, versus the unmangled names in the program.  Therefore, the
debugger can only look up mangled names versus original program names, which
makes debugging extremely difficult for programmers. To solve this program, the
language must provide the debugger with a "demangled" so it can convert
mangled names back to program names, and correspondingly retrieve the type of
the name. \cite{Reference9}

\CFA, a new language being developed at the University of Waterloo, has
overloading, so names resolved by the debugger are mangled names. Therefore,
another objective of this project is to add a \CFAS demangler to GDB.

% Name mangling is a technique used in compilers to resolve this
% problem. This technique provides a mechanism to encode additional information in the
% name of a function, or a variable to supply more semantic information from
% compiler to debugger \cite{Reference9}. \CFA, a new language being developed at the University of
% Waterloo, has overloading, so names resolved from
% the debugger are mangled names. As with early versions of \CC, it is not user-friendly to debug a program using
% mangled names. Therefore, another objective of this project is to add a \CFAS demangler in GDB.
