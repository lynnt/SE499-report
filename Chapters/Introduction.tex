\chapter{Introduction} \label{introduction}
Computer programming languages provide humans a means to instruct computers to
perform a particular task. New programming languages are
invented to simplify the task, or provide additional features enhance
performance, and improve developer productivity.

A crucial companion tool to a programming language is a debugger. A debugger is a productivity tool to aid developers in testing
and finding bugs in a program. By definition, a debugger executes
any program written in one of a supported set of languages and allows developers
to stop, monitor and examine state in the program for further investigation.

Specifically, this report talks about how to add GNU Debugger (GDB) support for the
programming languages \uCPPS and \CFA.
For \CFA, new hooks are also added to allow GDB to understand that
\CFAS is a new source language that requires invocation of a demangler for
variable and function names.

Because \uCPPS is a translator, all the code written in \uCPPS is eventually
compiled down to \CCS code. This transformation gives \uCPPS an advantage with
GDB because GDB already understands \CC. However, since \uCPPS introduced new objects
and high-level execution constructs into the language, GDB does not understand
these objects or the runtime environment. One objective of this
project is to write new GDB extensions to understand concurrency among tasks, a new high-level execution construct that is discussed more in Chapter \ref{uCPP}.

Additionally, if a programming language provides overloading functionality,
which is allowing functions or variables in the same scope with the
same identifier, then each of these entities must be assigned a unique name, otherwise,
there are name collisions.
Name mangling is a technique used in compilers to resolve this
problem. This technique provides a mechanism to encode additional information in the
name of a function, or a variable to supply more semantic information from
compiler to debugger \cite{reference9}. \CFA, a new language being developed at the University of
Waterloo, has overloading, so names resolved from
the debugger are mangled names. As with early versions of \CC, it is not user-friendly to debug a program using
mangled names. Therefore, another objective of this project is to add a \CFAS demangler in GDB.
