\chapter{Introduction} \label{introduction}

Computer programming languages allow humans to instruct computers to perform an
action in a language they understand. All these languages eventually translate
to binary, in which is a representation of number 0 and 1. New languages are
invented to perform a particular task better, or to be a replacement with
additional features and performance that existing language cannot do as well or
to improve developers' productivity.
\cite{Reference1}.

In the same way, debugger is a productivity tool to aid developers in testing
and finding bugs in their program. By definition, debugger is a computing program that runs
any programm written in one of the supported languages and allows developers to
monitor, examine states and stops the program for further investigation. In
particular, this report talks about how to add GNU debugger support for \uCPPS
and \CFA.

\section{Problem Statement}
Because \uCPPS is a translator, so all the code written in \uCPPS is eventually
compiled down to \CC. Additionally, GDB has been supporting \CCS for a long time,
so most of the work is already done by GNU developers. However, since \uCPPS introduced
new objects into the language to extend the existing models, GDB has yet learned
to understand these objects. Therefore, it is not possible to examine the state
of these objects and switch between tasks. One objective of this research
project is to extend GDB to allow perform context switch
between tasks and examine information about structures that only exist in \uCPPS
such as clusters, processors, etc.

In addition to extending GDB for \uCPPS, another objective of this research project
is to add GDB support for \CFA, which is a new language developing by the
University of Waterloo Programming Language group. In particular, the project
adds a new demangler for \CFAS because even though \CFAS is compiled
down to C, it allows overloading for both variables and functions, which is
non-existent in C.
