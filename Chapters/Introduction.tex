\chapter{Introduction} \label{introduction}

Computer programming languages allow humans to instruct computers to perform an
action in a language they understand. All these languages eventually translate
to binary, which is a representation of number 0 and 1. New languages are
invented to perform a particular task better, or to be a replacement with
additional features and performance that existing language cannot do as well or
to improve developers' productivity.
\cite{Reference1}.

In the same way, debugger is a productivity tool to aid developers in testing
and finding bugs in their program. By definition, debugger is a computing program that runs
any programm written in one of the supported languages and allows developers to
monitor, examine states and stops the program for further investigation. In
particular, this report talks about how to add GNU debugger support for \uCPPS
and \CFA.

\section{Problem Statement}
Because \uCPPS is a translator, so all the code written in \uCPPS is eventually
compiled down to \CC. Additionally, GDB has been supporting \CCS for a long time,
so most of the work is already done by GNU developers. However, since \uCPPS introduced
new objects into the language to extend the existing models, GDB has yet learned
to understand these objects. Therefore, it is not possible to examine the state
of these objects and switch between tasks. One objective of this research
project is to extend GDB to allow perform context switch
between tasks and examine information about structures that only exist in \uCPPS
such as clusters, processors, etc.

In compiler construction, the linker combines one or more object files generated
by compilers into a single executable file. However, if a programming
language allows overloading, defining different routines or variables with the
same identifier, each of these entities must be unique before linkers
combine all the object files, because otherwise, there is a collision in name.
The technique used by compilers to resolve collision is callled name mangling.
This technique provides a mechanism to encode additional information in the
name of a function, class or variable to supply more semantic information from
compilers to linkers \cite{reference9}. \CFA, a new language developed by the University of
Waterloo Programming Language Group, has overloading feature, so names resolved from
linkers are mangled name. It is inconvenient to debug using mangled name in
GDB. Therefore, another objective of this research is to add GDB support for \CFAS by
adding a demangler in GDB for \CFA.
