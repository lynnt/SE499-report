\chapter{Introduction} \label{introduction}
Computer programming languages allow humans to instruct computers to perform an
action in a language they understand. All these languages eventually translate
to binary, which is a representation of number 0 and 1. New languages are
invented to perform a particular task better, or to be a replacement with
additional features and performance improvement that existing languages cannot do as well or
improve developers' productivity\cite{Reference1}.

In the same way, debugger is a productivity tool to aid developers in testing
and finding bugs in their program. By definition, a debugger executes
any program written in one of the supported languages and allows developers to
monitor, examine states and stops the program for further investigation.

Specifically, this report talks about how to add GNU Debugger support for \uCPPS
and \CFA.
For \CFA, new hooks should be added to allow GDB understand
\CFAS as a new source language and invocation of a demangler for \CFA.

Because \uCPPS is a translator, all the code written in \uCPPS is eventually
compiled down to \CC's code. This transformation gives \uCPPS an advantage with
GDB because GDB understands \CC. However, since \uCPPS introduced new objects
and high-level execution constructs into the language to extend the existing
model, GDB has yet learned to understand these objects. One objective of this
project is to write new extensions to allow context switching
between tasks, a new high-level execution construct that is going to be discussed more in
details in Chapter \ref{uCPP}.

Additionally, in compiler construction, a linker combines one or more object files generated
by a compiler into a single executable file. However, if a programming
language provides overloading functionality, which is a permission for defining different routines or variables with the
same identifier, then each of these entities must be unique because otherwise,
there is going to be a name collision when a linker tries to combine all the
object files. Name mangling is the technique used in compilers to resolve this
problem. This technique provides a mechanism to encode additional information in the
name of a function, class or variable to supply more semantic information from
compilers to linkers\cite{reference9}. \CFA, a new language developed at the University of
Waterloo, has overloading feature, so names resolved from
linkers are mangled name. It is not user-friendly to debug using mangled name in
a debugger. Therefore, another objective of this research is to add a \CFAS demangler in GDB.
