\chapter{Introduction} \label{introduction}

Computer programming languages allow human to instruct computers to perform an
action in a language they understand. All these languages eventually translate
to binary, in which is a representation of number 0 and 1. New languages are
invented to perform a particular task better, or to be a replacement with
additional features and performance that existing language cannot do as well
\cite{Reference1}.

\section{Background}
This report talks about how to add GNU debugger support for \uCPPS and \CFA.

\section{Problem Statement}

This report discusses about adding GDB support for two different languages.

First of all, since \uCPPS is a translator, so all the code written in \uCPPS is eventually
compiled down to \CC. Additionally, GDB has been supporting \CCS for a long time,
so most of the work is already done by GNU developers. However, since \uCPPS introduced
new objects into the language to extend the existing models, GDB has yet learned
to understand these objects. Therefore, it is not possible to switch between
tasks. One objective of my research project is to extend GDB to allow switching
between tasks and examine information about structures that only exist in \uCPPS
such as clusters, processors, etc.

In addition to extending GDB for \uCPPS, another objective of my research project
is to add GDB support for \CFA, which is a new language developing by the
University of Waterloo Programming Language group. In particular, the project
adds a new demangler for \CFAS because even though \CFAS is compiled
down to C, it allows overloading for both variables and functions, which is
non-existent in C.
