%% ----------------------------------------------------------------
%% Thesis.tex -- MAIN FILE (the one that you compile with LaTeX)
%% ---------------------------------------------------------------- 

% Set up the document
\documentclass[legalpaper, 11pt, oneside]{Thesis}  % Use the "Thesis" style, based on the ECS Thesis style by Steve Gunn
\graphicspath{Figures/}  % Location of the graphics files (set up for graphics to be in PDF format)

% Include any extra LaTeX packages required
\usepackage[square, numbers, comma, sort&compress]{natbib}  % Use the "Natbib" style for the references in the Bibliography
\usepackage{verbatim}  % Needed for the "comment" environment to make LaTeX comments
\usepackage{vector}  % Allows "\bvec{}" and "\buvec{}" for "blackboard" style bold vectors in maths
\usepackage[pagewise]{lineno}
\usepackage{listings}
\hypersetup{urlcolor=blue, colorlinks=true}  % Colours hyperlinks in blue, but this can be distracting if there are many links.
\usepackage{geometry}
\geometry{
      letterpaper,
        left=1.5in,top=1in,right=1.0in,bottom=1in,
          headheight=20pt,headsep=0.25in,foot=9pt,footsep=0.3in,
            includeheadfoot
}
%% ----------------------------------------------------------------
\begin{document}
\frontmatter      % Begin Roman style (i, ii, iii, iv...) page numbering
\linenumbers    % TODO: remove this please
% Set up the Title Page
\title{GNU Debugger support for \uCPPS and \CFA}
\authors  {\texorpdfstring
            {\href{tmltran@uwaterloo.ca}{Thi My Linh Tran}}
            {Thi My Linh Tran}
            }
\emails{tmltran@uwaterloo.ca}
\addresses  {\groupname\\\deptname\\\univname}  % Do not change this here, instead these must be set in the "Thesis.cls" file, please look through it instead
\date       {\today}
\subject    {}
\keywords   {}
\maketitle
%% ----------------------------------------------------------------

\setstretch{1.3}  % It is better to have smaller font and larger line spacing than the other way round

% Define the page headers using the FancyHdr package and set up for one-sided printing
\fancyhead{}  % Clears all page headers and footers
\rhead{\thepage}  % Sets the right side header to show the page number
\lhead{}  % Clears the left side page header

\pagestyle{fancy}  % Finally, use the "fancy" page style to implement the FancyHdr headers

%% ----------------------------------------------------------------
% Declaration Page required for the Thesis, your institution may give you a different text to place here
\iffalse
\Declaration{

\addtocontents{toc}{\vspace{1em}}  % Add a gap in the Contents, for aesthetics
\end{itemize}
 
 
Signed:\\
\rule[1em]{25em}{0.5pt}  % This prints a line for the signature
 
Date:\\
\rule[1em]{25em}{0.5pt}  % This prints a line to write the date
}
\clearpage  % Declaration ended, now start a new page
\fi

%% ----------------------------------------------------------------
% The "Funny Quote Page"
\iffalse
\pagestyle{empty}  % No headers or footers for the following pages

\null\vfill
% Now comes the "Funny Quote", written in italics
\textit{``Write a funny quote here.''}

\begin{flushright}
If the quote is taken from someone, their name goes here
\end{flushright}

\vfill\vfill\vfill\vfill\vfill\vfill\null
\clearpage  % Funny Quote page ended, start a new page
\fi
%% ----------------------------------------------------------------
% The Abstract Page
\addtotoc{Preface}  % Add the "Abstract" page entry to the Contents
\preface{
\addtocontents{toc}{\vspace{1em}}  % Add a gap in the Contents, for aesthetics

The goal of this work is to add GNU Debugger support for the language \uCPPS and \CFA. To
achieve this goal for \uCPP, new extensions are written to provide additional
support for high-level constructs that are unknown to the GNU Debugger. In addition to the work done for
\uCPP, many hooks are added in the GNU Debugger to enable the addition of a \CFAS
demangler.

This report assumes the reader has basic knowledge of compiler construction.
Background knowledge about how GNU Debugger works and specific features of \uCPPS and
\CFAS
are provided in the report.
}

\clearpage  % Abstract ended, start a new page
%% ----------------------------------------------------------------

\setstretch{1.3}  % Reset the line-spacing to 1.3 for body text (if it has changed)

% The Acknowledgements page, for thanking everyone
\acknowledgements{
\addtocontents{toc}{\vspace{1em}}  % Add a gap in the Contents, for aesthetics

I would like to thank Professor Peter Buhr and Thierry
Delisle for their guidance throughout the development of this project. The
GNU Debugger's internal manual \cite{reference5} is used as a guide for the development of
adding a demangler for \CFA, and \CFA's main page \cite{reference13} for
background knowledge and examples.
}
\clearpage  % End of the Acknowledgements
%% ----------------------------------------------------------------

\pagestyle{fancy}  %The page style headers have been "empty" all this time, now use the "fancy" headers as defined before to bring them back


%% ----------------------------------------------------------------
\lhead{\emph{Contents}}  % Set the left side page header to "Contents"
\addcontentsline{toc}{chapter}{Listings}
\tableofcontents  % Write out the Table of Contents

%% ----------------------------------------------------------------
%\lhead{\emph{List of Figures}}  % Set the left side page header to "List if Figures"
%\listoffigures  % Write out the List of Figures

%% ----------------------------------------------------------------
%\lhead{\emph{List of Tables}}  % Set the left side page header to "List of Tables"
%\listoftables  % Write out the List of Tables

%% ----------------------------------------------------------------
\lhead{\emph{List of Listings}}  % Set the left side page header to "List of Listings"
\lstlistoflistings

%% ----------------------------------------------------------------
\setstretch{1.5}  % Set the line spacing to 1.5, this makes the following tables easier to read
%\clearpage  % Start a new page
%\lhead{\emph{Abbreviations}}  % Set the left side page header to "Abbreviations"
%\listofsymbols{ll}  % Include a list of Abbreviations (a table of two columns)
%{
% \textbf{Acronym} & \textbf{W}hat (it) \textbf{S}tands \textbf{F}or \\
%\textbf{LAH} & \textbf{L}ist \textbf{A}bbreviations \textbf{H}ere \\
%\textbf{GDB} & GNU Debugger \\
%\textbf{API} & Application Programming Interface \\
%}

%% ----------------------------------------------------------------
\clearpage  % Start a new page
\iffalse
%\lhead{\emph{Physical Constants}}  % Set the left side page header to "Physical Constants"
%\listofconstants{lrcl}  % Include a list of Physical Constants (a four column table)
{
% Constant Name & Symbol & = & Constant Value (with units) \\

}

%% ----------------------------------------------------------------
\clearpage  %Start a new page
\lhead{\emph{Symbols}}  % Set the left side page header to "Symbols"
\listofnomenclature{lll}  % Include a list of Symbols (a three column table)
{
% symbol & name & unit \\
}
\fi
%% ----------------------------------------------------------------
% End of the pre-able, contents and lists of things
% Begin the Dedication page

\setstretch{1.3}  % Return the line spacing back to 1.3

\addtocontents{toc}{\vspace{2em}}  % Add a gap in the Contents, for aesthetics


%% ----------------------------------------------------------------
\mainmatter	  % Begin normal, numeric (1,2,3...) page numbering
\pagestyle{fancy}  % Return the page headers back to the "fancy" style

% Include the chapters of the thesis, as separate files
% Just uncomment the lines as you write the chapters

\lhead{\emph{Introduction}}  % Set the left side page header to "Abbreviations"
\chapter{Introduction} \label{introduction}
Computer programming languages provide humans a means to instruct computers to
perform a particular task. New programming languages are
invented to simplify the task, or provide additional features enhance
performance, and improve developer productivity.

A crucial companion tool to a programming language is a debugger. A debugger is a productivity tool to aid developers in testing
and finding bugs in a program. By definition, a debugger executes
any program written in one of a supported set of languages and allows developers
to stop, monitor and examine state in the program for further investigation.

Specifically, this report talks about how to add GNU Debugger (GDB) support for the
programming languages \uCPPS and \CFA.
For \CFA, new hooks are also added to allow GDB to understand that
\CFAS is a new source language that requires invocation of a demangler for
variable and routine names.

Because \uCPPS is a translator, all the code written in \uCPPS is eventually
compiled down to \CCS code. This transformation gives \uCPPS an advantage with
GDB because GDB already understands \CC. However, since \uCPPS introduced new objects
and high-level execution constructs into the language, GDB does not understand
these objects or the runtime environment. One objective of this
project is to write new GDB extensions to understand concurrency among tasks, a new high-level execution construct that is discussed more in Chapter \ref{uCPP}.

Additionally, if a programming language provides overloading functionality,
which allowing routines or variables in the same scope with the
same identifier, then each of these entities must be assigned a unique name, otherwise,
there are name collisions.
Name mangling is a technique used in compilers to resolve this
problem. This technique provides a mechanism to encode additional information in the
name of a function, or a variable to supply more semantic information from
compiler to debugger \cite{reference9}. \CFA, a new language being developed at the University of
Waterloo, has overloading, so names resolved from
the debugger are mangled names. As with early versions of \CC, it is not user-friendly to debug a program using
mangled names. Therefore, another objective of this project is to add a \CFAS demangler in GDB.
 % Introduction

% Background
\clearpage  % Start a new page
\lhead{\emph{\uCPP}}  % Set the left side page header to "Abbreviations"
\chapter{\uCPP} \label{uCPP}

\section{Introduction}
\uCPPS \cite{reference10} extends the \CCS programming language with new
mechanisms to
facilitate control flow, and adds new objects to enable lightweight concurrency
on uniprocessor and parallel execution on multiprocessor computers. Concurrency has tasks
that can context switch, which is a control transfer from one execution state to
another that is different from the routine call. Tasks are selected to run on a
processor from a ready queue of available tasks, and tasks may need to wait for an occurrence of an event.

\section{Tasks}
A \textcolor{ForestGreen}{task} behaves like a class object, but it maintains its own
thread and execution state, which is the state information required to allow
independent execution. A task provides mutual exclusion by default for calls to its
public members. Public members allow communication among tasks.

\section{\uCPPS Runtime Structure}
\uCPPS introduces two new runtime entities for controlling concurrent execution:
\begin{itemize}
    \item Cluster
    \item Virtual processor
\end{itemize}

\subsection{Cluster}
A cluster is a group of tasks and virtual processors (discussed next) that
execute tasks. The objective of a cluster is to control the amount of possible
parallelism mong tasks, which is only feasible if there are many
multiprocessors.

At the start of a \uCPPS program, two clusters are created. One is the system
cluster and the other is the user cluster. The system cluster has a
processor that only performs management work such as error detection and
correction from user clusters, if an execution in a user cluster results in errors,
and cleans up at shutdown. The user cluster manages and
executes user tasks on its processors. The benefits of clusters
are maximized utilization of processors and miminization of runtime through a scheduler that is appropriate for a particular workload. Tasks and virtual
processors may be migrated among clusters.

\subsection{Virtual Processor}
A virtual processor is a software emulation of a processor that executes
threads. Kernel threads are used to implement a virtual processor, which are
scheduled for execution on a hardware processor by the underlying operating
system. The operating system distributes kernel threads across a number of
processors assuming that the program runs on a multiprocessor architecture. The
usage of kernel threads enables parallel execution in \uCPP.

\clearpage  %Start a new page
\lhead{\emph{GNU Debugger}}  % Set the left side page header to "Abbreviations"
\chapter{GNU Debugger}

\section{Mangling}
\section{Demangling}

\clearpage  %Start a new page
\lhead{\emph{\CFA}}  % Set the left side page header to "Abbreviations"
\chapter{\CFA} \label{CFA}
\section{Introduction}
\textbf{\textcolor{red}{TODO}}: rephrase all these stuff in my own word and cite

Similar to \CC, C is a popular programming language especially in systems such
as operating systems, embedded systems. For example, Windows NT kernel and
Linux kernel are written in C, and they are the foundation of many higher level
and popular projects. Therefore, it is unlikely that they will go away any time soon.

However, C has in syntactics, linkage, semantics and many
more. Even though \CCS is meant to fix these problems, \CCS has legacy design
choices that cannot be undone and newer versions of \CCS require significantly
more effort to convert C-based projects into \CCS.

To solve this problem, \CFAS is designed by the programming language group at the
University of Waterloo. The language's goal is to extend C with modern language
features that many new languages have been known for such as Rust, Go. This
extension provides a backward compatible version of C while fixing existing
problems known in C.

\section{Overloading}
Overloading is when a compiler that allows defining two different routines with
the same name, but there is a
In most programming languages, they only allow operator and/or function overloading.
In addition to those, \uCPP also supports variables and literal 0/1 overloading.

\subsection{Variables}
Variables in the same block are allowed to have the same name as long as it is
differentiated by type. An assignment of a new variable to the same variable is
deferred from type.
\begin{frame}
\frametitle{}
\lstset{language=C++,
        basicstyle=\ttfamily,
        keywordstyle=\color{blue}\ttfamily,
        stringstyle=\color{red}\ttfamily,
        commentstyle=\color{ForestGreen}\ttfamily,
        morecomment=[l][\color{magneta}]{\#}}
\begin{lstlisting}
short int MAX = 3;
int MAX = 4;
double MAX = 1.0;

// select variable MAX based on its left-hand type
short int s = MAX; // MAX = 3
int max = MAX; // MAX = 4
double max = MAX; // MAX = 1.0
\end{lstlisting}
\end{frame}

\subsection{Routine}
Routines in the same block can be overloaded depending on the number and type of
parameters and returns.
\begin{frame}
\frametitle{}
\lstset{language=C++,
        basicstyle=\ttfamily,
        keywordstyle=\color{blue}\ttfamily,
        stringstyle=\color{red}\ttfamily,
        commentstyle=\color{ForestGreen}\ttfamily,
        morecomment=[l][\color{magneta}]{\#}}
\begin{lstlisting}
void f(<@\textcolor{red}{void}@>);              // (1)
void f(<@\textcolor{red}{char}@>);              // (2)
<@\textcolor{red}{char}@> f(void);              // (3)
<@\textcolor{red}{[int,double]}@> f();          // (4)

f();                        // pick (1)
f('a');                     // pick (2)
char s = f('a');            // pick (3)
[int, double] s = f();      // pick (4)
\end{lstlisting}
\end{frame}

\subsection{Operator}
An operator name is denoted with \? for the operand and any standard C
operator. Operator names within the same block can be overloaded depending on
the number and type of parameters and returns. However, operators \textit{??},
\textit{||}, \textit{?:} cannot be overloaded because short-circuit semantics
cannot be preserved. This behaves as same as how \CCS behaves.
\begin{frame}
\frametitle{}
\lstset{language=C++,
        basicstyle=\ttfamily,
        keywordstyle=\color{blue}\ttfamily,
        stringstyle=\color{red}\ttfamily,
        commentstyle=\color{ForestGreen}\ttfamily,
        morecomment=[l][\color{magneta}]{\#}}
\begin{lstlisting}
int <@\texcolor{red}{++?}@>(int op);            // unary prefix increment
int <@\textcolor{red}{?++}@>(int op);           // unary postfix increment
int <@\textcolor{red}{?+?}@>(int op1, int op2); // unary postfix increment

struct A { double x, double y }

// overload operator plus-assignment
A ?+?(S a, S b) {
    return (S) {a.x + b.x, a.y + b.y};
}
\end{lstlisting}
\end{frame}

\clearpage  %Start a new page

% Real work done

\lhead{\emph{GNU Debugger Extensions for \uCPP}}  % Set the left side page header to "Abbreviations"
\chapter{Extending GDB for \uCPP}
\section{Design Constraints}
As mentioned in Chapter \ref{GDB}, there are many ways to extend GDB. However,
there are a few design constraints on the selected mechanism. Extension
mechanism for \uCPPS should be simple, easy to implement and versatile.

Regardless of which mechanism selected for extending GDB, all the functions
implemented should maintain the same functionality. In addition to functional
requirements, user usability and flexibility are additional required
non-functional requirements for this
project. These final requirements enable developers to be productive quickly
and can do more with the extensions.

\section{Design Implementation}
Firstly, Python API satifies all the given criteria above after researching
about different solutions. In particular, GDB provides an extensive API for
Python compared to other solutions. Furthermore, Python is a scripting language with built-in
data structures and functions that enables the development of more complex
operations and yet saves time on development of the project.

The next task for extending GDB is writing new user-defined commands that allow
users to switch from one task to another task. This means that there needs to be at
least two commands. The first command is for switching from the current task to
the new task that can be called \verb|push_task|, and another command
is for switching back to the previous task, from which the current task comes
from, and this command is named \verb|pop_task|.

Prior to implementing either \verb|pushtask| or \verb|poptask| command, many questions come up about how
these command should work.
\begin{itemize}
\item What format of the argument of the new task should it the command
\verb|pushtask| be?
\item Should \verb|pushtask| command allow users to pass the address of the desired task or can users
pass in an index of the cluster, in which the task is, along with the index of
the task with respect to that cluster?
\item Should \verb|pushtask| command save the context of the current task before it switches to
another task every time? Or should the command only save the context of the
originating task?
\end{itemize}

As a result, there are two variations of \verb|pushtask| command to provide both
user usability and flexibility. The first variation is called \verb|pushtask| and requires
only one argument, which is the address of the task a user wants to switch to.
The second variation is called \verb|pushtask_id| and requires both the index of
the cluster and the index of the task a user would like
to switch to as arguments. Core functionality of these two commands is
essentially the same. The only difference is that the \verb|pushtask_id| command
needs to iterate through the data structure that stores all the clusters to find
the right cluster based on its index, and using the found instance of the
cluster to search for the instance of the expected task. Additionally, the second
command invokes a function that performs the context switch between tasks, which
is shared by these two commands. Furthermore, a decision
is made to store the context information for every context switching, this means
that \verb|pushtask| command needs to perform actions to store these information every time it is
invoked. Each task has a copy of \verb|uContext_t|, which stores the context
information such as stack pointer, frame pointer. These pointers are then stored
in convenience variables for every level of redirection. Similarly, the command
also retrieves these context information but from the task that a user wants to switch to, and
sets the equivalent registers to the new values. In particular, the value of the stack
pointer is set in \verb|rsp| register, the value of the frame pointer is set in \verb|rbp|
register, and finally the value of the program counter is set in \verb|pc| register.
Normally, register values are relative to the current stack frame assuming that
all the stack frames farther already exited and their registers are restored. Therefore, in
order to see the true value of hardware registers, innermost frame that is
frame-0 must be selected\cite{reference11}. However, it is possible to not to be in frame-0, so prior to setting these values,
the command must switch back to the innermost (currently executing) frame first.

Finally, the implementation of \verb|poptask| is more straightforward and does
not require different variations of the command. This command goes back to the previous task, however, this may not be the original
task that initiates the task switching process. If the current context is
already in the starting task, the command does nothing. Otherwise,
\verb|poptask| command first switches back to the innermost frame, and then
retrieves the
address of the last task it was switched from, and sets the same set of registers as
\verb|pushtask| command did but to the values of the last task's context
information.

\section{Result}
The current implementation successfully allows users to switch from one task to
another task, but does not permit the switch if the destination task is an already terminated
task. However, if a user would like to continue the execution where it is left
off assuming that the program has yet crashed, then the user needs to perform
\verb|poptask| all the way back to the originating task before moving forward.

\clearpage  %Start a new page
\lhead{\emph{\CFAS Demangler}}  % Set the left side page header to "Abbreviations"
\chapter{\CFAS Demangler} \label{demangler}

\CFAS As A New Source Language in GDB

Adding \CFAS to DWARF reader

Symbol lookup for \CFAS

Adding new demangler

\clearpage  %Start a new page

\lhead{\emph{Conclusion}}  % Set the left side page header to "Abbreviations"
\chapter{Conclusion}
New extensions are written to support context switching between \uCCS user tasks, and
new hooks are added to the GNU
Debugger to support \CFAS demangler. GDB provides Python extension API and hooks to add debugging support for a
new source language. In particular, writing Python extensions is easier and more
robust. The usage of the Python language enables writing of
more complex operations with built-in data structures and functions.
Furthermore, GDB provides sufficient hooks to make it easier for a new language to leverage existing
infrastructure and codebase to add debugging support.
 % Conclusion

%% ----------------------------------------------------------------
% Now begin the Appendices, including them as separate files

\addtocontents{toc}{\vspace{2em}} % Add a gap in the Contents, for aesthetics

\appendix % Cue to tell LaTeX that the following 'chapters' are Appendices

% TODO: remove if want appendex
%\chapter{An Appendix}

Lorem ipsum dolor sit amet, consectetur adipiscing elit. Vivamus at pulvinar nisi. Phasellus hendrerit, diam placerat interdum iaculis, mauris justo cursus risus, in viverra purus eros at ligula. Ut metus justo, consequat a tristique posuere, laoreet nec nibh. Etiam et scelerisque mauris. Phasellus vel massa magna. Ut non neque id tortor pharetra bibendum vitae sit amet nisi. Duis nec quam quam, sed euismod justo. Pellentesque eu tellus vitae ante tempus malesuada. Nunc accumsan, quam in congue consequat, lectus lectus dapibus erat, id aliquet urna neque at massa. Nulla facilisi. Morbi ullamcorper eleifend posuere. Donec libero leo, faucibus nec bibendum at, mattis et urna. Proin consectetur, nunc ut imperdiet lobortis, magna neque tincidunt lectus, id iaculis nisi justo id nibh. Pellentesque vel sem in erat vulputate faucibus molestie ut lorem.

Quisque tristique urna in lorem laoreet at laoreet quam congue. Donec dolor turpis, blandit non imperdiet aliquet, blandit et felis. In lorem nisi, pretium sit amet vestibulum sed, tempus et sem. Proin non ante turpis. Nulla imperdiet fringilla convallis. Vivamus vel bibendum nisl. Pellentesque justo lectus, molestie vel luctus sed, lobortis in libero. Nulla facilisi. Aliquam erat volutpat. Suspendisse vitae nunc nunc. Sed aliquet est suscipit sapien rhoncus non adipiscing nibh consequat. Aliquam metus urna, faucibus eu vulputate non, luctus eu justo.

Donec urna leo, vulputate vitae porta eu, vehicula blandit libero. Phasellus eget massa et leo condimentum mollis. Nullam molestie, justo at pellentesque vulputate, sapien velit ornare diam, nec gravida lacus augue non diam. Integer mattis lacus id libero ultrices sit amet mollis neque molestie. Integer ut leo eget mi volutpat congue. Vivamus sodales, turpis id venenatis placerat, tellus purus adipiscing magna, eu aliquam nibh dolor id nibh. Pellentesque habitant morbi tristique senectus et netus et malesuada fames ac turpis egestas. Sed cursus convallis quam nec vehicula. Sed vulputate neque eget odio fringilla ac sodales urna feugiat.

Phasellus nisi quam, volutpat non ullamcorper eget, congue fringilla leo. Cras et erat et nibh placerat commodo id ornare est. Nulla facilisi. Aenean pulvinar scelerisque eros eget interdum. Nunc pulvinar magna ut felis varius in hendrerit dolor accumsan. Nunc pellentesque magna quis magna bibendum non laoreet erat tincidunt. Nulla facilisi.

Duis eget massa sem, gravida interdum ipsum. Nulla nunc nisl, hendrerit sit amet commodo vel, varius id tellus. Lorem ipsum dolor sit amet, consectetur adipiscing elit. Nunc ac dolor est. Suspendisse ultrices tincidunt metus eget accumsan. Nullam facilisis, justo vitae convallis sollicitudin, eros augue malesuada metus, nec sagittis diam nibh ut sapien. Duis blandit lectus vitae lorem aliquam nec euismod nisi volutpat. Vestibulum ornare dictum tortor, at faucibus justo tempor non. Nulla facilisi. Cras non massa nunc, eget euismod purus. Nunc metus ipsum, euismod a consectetur vel, hendrerit nec nunc.	% Appendix Title

%\input{Appendices/AppendixB} % Appendix Title

%\input{Appendices/AppendixC} % Appendix Title

\addtocontents{toc}{\vspace{2em}}  % Add a gap in the Contents, for aesthetics
\backmatter

%% ----------------------------------------------------------------
\label{Bibliography}
\lhead{\emph{Bibliography}}  % Change the left side page header to "Bibliography"
\bibliographystyle{unsrtnat}  % Use the "unsrtnat" BibTeX style for formatting the Bibliography
\bibliography{Bibliography}  % The references (bibliography) information are stored in the file named "Bibliography.bib"

\end{document}  % The End
%% ----------------------------------------------------------------
